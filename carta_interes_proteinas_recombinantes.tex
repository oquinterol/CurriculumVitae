%%%%%%%%%%%%%%%%%%%%%%%%%%%%%%%%%%%%%%%%%
% Awesome Cover Letter
% XeLaTeX Template
% Version 1.3 (30/3/2020)
%
% This template has been downloaded from:
% http://www.LaTeXTemplates.com
%
% Original authors:
% Claud D. Park (posquit0.bj@gmail.com)
% Lars Richter (mail@ayeks.de)
% With modifications by:
% Vel (vel@latextemplates.com)
%
% License:
% CC BY-NC-SA 3.0 (http://creativecommons.org/licenses/by-nc-sa/3.0/)
%
% Important note:
% This template must be compiled with XeLaTeX, the below lines will ensure this
%!TEX TS-program = xelatex
%!TEX encoding = UTF-8 Unicode
%
%%%%%%%%%%%%%%%%%%%%%%%%%%%%%%%%%%%%%%%%%

%----------------------------------------------------------------------------------------
%	PACKAGES AND OTHER DOCUMENT CONFIGURATIONS
%----------------------------------------------------------------------------------------

\documentclass[11pt, letterpaper]{awesome-cv} % A4 paper size by default, use 'letterpaper' for US letter

\geometry{left=2cm, top=1.5cm, right=2cm, bottom=2cm, footskip=.5cm} % Configure page margins with geometry

\fontdir[fonts/] % Specify the location of the included fonts

% Color for highlights
\colorlet{awesome}{awesome-red} % Default colors include: awesome-emerald, awesome-skyblue, awesome-red, awesome-pink, awesome-orange, awesome-nephritis, awesome-concrete, awesome-darknight
%\definecolor{awesome}{HTML}{CA63A8} % Uncomment if you would like to specify your own color

% Colors for text - uncomment and modify
%\definecolor{darktext}{HTML}{414141}
%\definecolor{text}{HTML}{414141}
%\definecolor{graytext}{HTML}{414141}
%\definecolor{lighttext}{HTML}{414141}

\renewcommand{\acvHeaderSocialSep}{\quad\textbar\quad} % If you would like to change the social information separator from a pipe (|) to something else

%----------------------------------------------------------------------------------------
%	PERSONAL INFORMATION
%	Comment any of the lines below if they are not required
%----------------------------------------------------------------------------------------

\name{Oscar A.}{Quintero López}
%\address{}
%\mobile{(+57) 319 623 0827}

\email{oquinterol@unal.edu.co}
\homepage{quinterol.com}
\github{oquinterol}
\linkedin{quinterol}
\twitter{@oquinterol}

\position{Técnico de Laboratorios de Biología - Bioinformático} % Your expertise/fields
%\quote{``Make the change that you want to see in the world."} % A quote or statement

%----------------------------------------------------------------------------------------
%	RECIPIENT/POSITION/LETTER INFORMATION
%	All of the below lines must be filled out
%----------------------------------------------------------------------------------------

\recipient{Dr. Edwin F. Sánchez López}{Grupo de Investigación en Bioquímica y Biología Molecular\\Universidad Distrital Francisco José de Caldas\\Bogotá, Colombia} % The department or recipient

\letterdate{\today} % The date on the letter, default is the date of compilation

\lettertitle{Carta de Interés - Curso Producción de Proteínas Recombinantes} % The title of the letter

\letteropening{Estimado Dr. Sánchez López,} % How the letter is opened

\letterclosing{Atentamente,} % How the letter is closed

\letterenclosure[Adjunto]{Currículum Vitae} % Any enclosures with the letter

\makecvfooter{\today}{Oscar A. Quintero López~~~·~~~Carta de Interés}{\thepage} % Specify the letter footer with 3 arguments: (<left>, <center>, <right>), leave any of these blank if they are not needed

%----------------------------------------------------------------------------------------

\begin{document}

\makecvheader % Print the header

\makelettertitle % Print the title

%----------------------------------------------------------------------------------------
%	LETTER CONTENT
%----------------------------------------------------------------------------------------

\begin{cvletter}

%------------------------------------------------

\lettersection{Sobre mí}

Soy Oscar A. Quintero López, Licenciado en Biología con formación complementaria en Bioinformática, y actualmente curso una maestría en Bioinformática en la Universidad Nacional de Colombia, donde desarrollo mi investigación como parte del grupo GiBBS (Bioinformatics and Computational Systems Biology Research Group) del Instituto de Genética. Paralelamente, me desempeño como Técnico de Laboratorios de Biología en la Universidad Distrital Francisco José de Caldas. Mi trayectoria académica se ha centrado en la genética de plantas, fitomejoramiento y la aplicación de modelos computacionales y herramientas bioinformáticas para el análisis de datos biológicos.

%------------------------------------------------

\lettersection{Motivación e Interés}

Mi interés en el curso de Producción de Proteínas Recombinantes surge de la necesidad de fortalecer mis competencias en biotecnología molecular desde una perspectiva tanto computacional como experimental. Durante mi formación en el grupo GiBBS, he desarrollado habilidades en análisis de datos genómicos, modelado computacional y diseño in silico, pero reconozco que comprender a profundidad los procesos de diseño, producción y caracterización de proteínas recombinantes es fundamental para cerrar la brecha entre el análisis bioinformático y la aplicación biotecnológica.

Este curso representa una oportunidad invaluable para ganar experiencia práctica en técnicas experimentales de laboratorio que complementen mi formación computacional. El componente práctico del curso, que incluye transformación de bacterias, extracción de ADN plasmídico, PCR, electroforesis y análisis de proteínas, me permitirá desarrollar competencias experimentales que fortalecerán mi perfil como investigador, integrando el trabajo de bioinformática con la validación experimental.

El enfoque integral del curso, que abarca desde el diseño molecular hasta las aplicaciones biotecnológicas en diferentes sistemas de expresión (procariotas y eucariotas, incluyendo plantas), se alinea perfectamente con mis intereses de investigación en genética vegetal y fitomejoramiento. Particularmente, las sesiones sobre diseño computacional de proteínas, búsqueda y predicción de genes, modelado estructural, molecular farming y edición génica con CRISPR representan áreas de gran relevancia para mi desarrollo profesional y complementan directamente mi formación en bioinformática y biología de sistemas.

%------------------------------------------------

\lettersection{Aporte al Curso}

Mi formación en bioinformática y biología de sistemas me permite aportar una perspectiva computacional al análisis de secuencias, diseño de vectores y modelado estructural de proteínas. Como parte del grupo GiBBS, he desarrollado competencias en el uso de herramientas bioinformáticas para el análisis genómico y la predicción de estructuras, lo cual complementa los aspectos teóricos del curso relacionados con diseño molecular, búsqueda en bases de datos (NCBI, GenBank, UniProt) y análisis de marcos de lectura abiertos.

Considero que mi perspectiva desde la bioinformática enriquecerá las discusiones del curso, especialmente en las sesiones relacionadas con diseño computacional, optimización de genes mediante análisis de codones, y el uso de inteligencia artificial en el diseño de proteínas. Además, mi participación en proyectos de investigación en genética vegetal me permite visualizar aplicaciones directas del conocimiento adquirido en este curso, tanto en mi investigación de maestría como en mi desarrollo profesional.

%------------------------------------------------

\end{cvletter}

%----------------------------------------------------------------------------------------

\makeletterclosing % Print the signature and enclosures

\end{document}
