%%%%%%%%%%%%%%%%%%%%%%%%%%%%%%%%%%%%%%%%%
% Awesome Cover Letter
% XeLaTeX Template
% Version 1.3 (30/3/2020)
%
% This template has been downloaded from:
% http://www.LaTeXTemplates.com
%
% Original authors:
% Claud D. Park (posquit0.bj@gmail.com)
% Lars Richter (mail@ayeks.de)
% With modifications by:
% Vel (vel@latextemplates.com)
%
% License:
% CC BY-NC-SA 3.0 (http://creativecommons.org/licenses/by-nc-sa/3.0/)
%
% Important note:
% This template must be compiled with XeLaTeX, the below lines will ensure this
%!TEX TS-program = xelatex
%!TEX encoding = UTF-8 Unicode
%
%%%%%%%%%%%%%%%%%%%%%%%%%%%%%%%%%%%%%%%%%

%----------------------------------------------------------------------------------------
%	PACKAGES AND OTHER DOCUMENT CONFIGURATIONS
%----------------------------------------------------------------------------------------

\documentclass[11pt, letterpaper]{awesome-cv} % A4 paper size by default, use 'letterpaper' for US letter

\geometry{left=2cm, top=1.5cm, right=2cm, bottom=2cm, footskip=.5cm} % Configure page margins with geometry
 
\fontdir[fonts/] % Specify the location of the included fonts

% Color for highlights
\colorlet{awesome}{awesome-red} % Default colors include: awesome-emerald, awesome-skyblue, awesome-red, awesome-pink, awesome-orange, awesome-nephritis, awesome-concrete, awesome-darknight
%\definecolor{awesome}{HTML}{CA63A8} % Uncomment if you would like to specify your own color

% Colors for text - uncomment and modify
%\definecolor{darktext}{HTML}{414141}
%\definecolor{text}{HTML}{414141}
%\definecolor{graytext}{HTML}{414141}
%\definecolor{lighttext}{HTML}{414141}

\renewcommand{\acvHeaderSocialSep}{\quad\textbar\quad} % If you would like to change the social information separator from a pipe (|) to something else

%----------------------------------------------------------------------------------------
%	PERSONAL INFORMATION
%	Comment any of the lines below if they are not required
%----------------------------------------------------------------------------------------

\name{Oscar A.}{Quintero López}
%\address{}
%\mobile{(+57) 319 623 0827}

\email{oquinterol@unal.edu.co}
\homepage{quinterol.com}
\github{oquinterol}
\linkedin{quinterol}
\twitter{@oquinterol}

\position{Licenciado en Biología - Bioinformático} % Your expertise/fields
%\quote{``Make the change that you want to see in the world."} % A quote or statement

%----------------------------------------------------------------------------------------
%	RECIPIENT/POSITION/LETTER INFORMATION
%	All of the below lines must be filled out
%----------------------------------------------------------------------------------------

\recipient{Departamento de Agronomía}{Universidad Nacional de Colombia\\Bogotá, Colombia} % The department or recipient

\letterdate{\today} % The date on the letter, default is the date of compilation

\lettertitle{Solicitud de Beca de Asistente Docente} % The title of the letter

\letteropening{Estimados miembros del comité,} % How the letter is opened

\letterclosing{Atentamente,} % How the letter is closed

\letterenclosure[Adjunto]{Currículum Vitae} % Any enclosures with the letter

\makecvfooter{\today}{Oscar A. Quintero López~~~·~~~Cover Letter}{\thepage} % Specify the letter footer with 3 arguments: (<left>, <center>, <right>), leave any of these blank if they are not needed
  
%----------------------------------------------------------------------------------------

\begin{document}

\makecvheader % Print the header

\makelettertitle % Print the title

%----------------------------------------------------------------------------------------
%	LETTER CONTENT
%----------------------------------------------------------------------------------------

\begin{cvletter}

%------------------------------------------------

\lettersection{Sobre mí}

Soy Oscar A. Quintero López, Licenciado en Biología con formación en Bioinformática, y actualmente curso una maestría en Bioinformática en la Universidad Nacional de Colombia en colaboración con el grupo de investigación BIOMOLc de la Universidad Distrital Francisco José de Caldas. Mi trayectoria académica y de investigación se ha centrado en la genética de plantas y fitomejoramiento, áreas en las que he desarrollado tanto mi trabajo de pregrado como mi investigación de maestría.

%------------------------------------------------

\lettersection{Motivación}

Mi interés principal radica en la aplicación de herramientas computacionales avanzadas en biología, con un enfoque en la mejora genética de cultivos. Durante mi tiempo en el grupo de investigación BIOMOLc, he contribuido en proyectos que integran técnicas de biología molecular y bioinformática para optimizar procesos de fitomejoramiento, utilizando modelos computacionales para predecir y seleccionar rasgos agronómicos deseables. Este enfoque es crucial para abordar los desafíos actuales en la agricultura, especialmente en el contexto de la seguridad alimentaria y la sostenibilidad.

%------------------------------------------------

\lettersection{Por qué en el Departamento de Agronomía}

La Universidad Nacional de Colombia, y en particular el Departamento de Agronomía, ofrece un entorno académico y de investigación idóneo para continuar con mi desarrollo profesional y académico. Estoy convencido de que mi experiencia y habilidades no solo contribuirán a las actividades de formación del curso de Genética General, sino que también aportarán valor en proyectos de investigación colaborativos que aborden los retos del fitomejoramiento mediante el uso de herramientas bioinformáticas.

%------------------------------------------------

\lettersection{Conclusión}

Estoy muy interesado en formar parte del equipo docente del Departamento de Agronomía como asistente, y en contribuir con mi experiencia en investigación en genética de plantas y mi conocimiento en bioinformática para el desarrollo de proyectos académicos y científicos. Agradezco la consideración de mi solicitud y quedo a disposición para cualquier información adicional que se requiera.

%------------------------------------------------

\end{cvletter}

%----------------------------------------------------------------------------------------

\makeletterclosing % Print the signature and enclosures

\end{document}
