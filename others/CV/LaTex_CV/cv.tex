\documentclass[11pt,letterpaper]{moderncv}

% Paquetes necesarios
% \usepackage{ifthen}
% \usepackage{ifpdf}
% \usepackage{color}
\usepackage{pifont}
%\usepackage{lmodern} Permite el uso de multiples tamaños de fuentes
%\usepackage{marvosym}
\usepackage{url}
% \usepackage{hyperref}
% \usepackage{longtable}
% \usepackage{graphicx}
% \usepackage{fancyhdr}

% Codificación texto
\usepackage[utf8]{inputenc}

% Ajuste de margenes
\usepackage[margin=1in]{geometry}
\recomputelengths

% Temas
\moderncvtheme[blue]{classic}

% Cambiar símbolo de fax para numero de identificación
\renewcommand{\mobilesymbol}{\ding{44}}
% Información personal
\firstname{Oscar Alexis}
\familyname{Quintero López}
\title{Resumen}
\extrainfo{C.C. 1069265957}
\phone{3196230827}
\email{oaquinterol@correo.udistrital.edu.co}
%\email{oaql97@gmail.com}

\nopagenumbers{}

\begin{document}
    \maketitle
    
    \section{Experiencia}
        \cventry{2019}{Tallerista}{Universidad Distrital Francisco José de Caldas}{Bogotá D.C.}{Regional Training Course on Gene Expression Anaysis Using RNA Seq Technology for Genetic Improvement of Mutant Crops}{}
        \cventry{2018}{Tallerista}{Universidad Distrital Francisco José de Caldas}{Bogotá D.C.}{}{Seminario-Taller: La construcción del significado del concepto de campo vectorial en programas de ingeniería}
        
        \subsection{Practica}
        \cventry{2020}{Practicas Docentes}{Universidad Distrital Francisco José de Caldas}{Bogotá D.C.}{Colegio Instituto Técnico Industrial Francisco José De Caldas}{}
    
    % \section{Proyectos}
    %     \cventry{2022}{Implementación protocolo LoRaWAN}{Universidad Nacional de Colombia}{Bogotá D.C.}{}{}

    \section{Educación}
        %\cventry{year--year}{Degree}{Institution}{City}{\textit{Grade}}{Description}
        \cventry{2016--2022}{Licenciatura en Biología}{Universidad Distrital Francisco José de Caldas}{Bogotá D.C.}{}{Trabajo de Grado: Desarrollo de flujo de trabajo (pipeline) para análisis de mutantes sólidos de \textit{Solanum tuberosum Gr. Phureja}}
    
    \section{Educación Complementaria}
        \cventry{2021}{Curso}{Programación Basica}{Bogotá D.C.}{Platzi}{}
        \cventry{2021}{Curso}{Configuración Entorno de Desarrollo en Linux}{Bogotá D.C.}{Platzi}{}
        \cventry{2021}{Curso}{Introduccion a Terminal y Linea de Comandos}{Bogotá D.C.}{Platzi}{}
        \cventry{2021}{Curso}{Astrobiología}{Bogotá D.C.}{Platzi}{}
        \cventry{2019}{Workshop}{Mechanical Forces in Biology: theory and simulations}{Bogotá D.C.}{}{}    

    \section{Idiomas}
        \cvlanguage{Inglés}{B1}{}

    \section{Habilidades}
        \cvcomputer{Programación}{Python, R/Bioconductor, Java, C/C++, Bash, \LaTeX}{Compilación}{CMake}
        \cvcomputer{Versionado}{git (GitHub, GitLab)}{Plataformas}{Linux, Docker}
        \cvcomputer{Tecnologias}{Snakemake, FastQC, SAMtools, Bowtie2, BWA}{}{}
        

% \section{Master thesis}
% \cvline{title}{\emph{Title}}
% \cvline{supervisors}{Supervisors}
% \cvline{description}{\small Short thesis abstract}
% %
% \section{Experience}
% \subsection{Vocational}
% \cventry{year--year}{Job title}{Employer}{City}{}{Description}
% \cventry{year--year}{Job title}{Employer}{City}{}{Description}
% \subsection{Miscellaneous}
% \cventry{year--year}{Job title}{Employer}{City}{}%
% {Description line 1\newline{}Description line 2}
% %
% \section{Languages}
% \cvlanguage{language 1}{Skill level}{Comment}
% \cvlanguage{language 2}{Skill level}{Comment}
% %
% \section{Computer skills}
% \cvcomputer{category 1}{XXX, YYY, ZZZ}{category 3}{XXX, YYY, ZZZ}
% \cvcomputer{category 2}{XXX, YYY, ZZZ}{category 4}{XXX, YYY, ZZZ}
% %
% \section{Interests}
% \cvline{hobby 1}{\small Description}
% \cvline{hobby 2}{\small Description}
% \cvline{hobby 3}{\small Description}
% %
% \closesection{}
% \pagebreak
% %
% \section{Extra}
% \cvlistitem{Item 1}
% \cvlistitem{Item 2}
% \cvlistitem{Item 3}
% %
% \section{Extra 2}
% \cvlistdoubleitem{Item 1}{Item 4}
% \cvlistdoubleitem{Item 2}{Item 5}
% \cvlistdoubleitem{Item 3}{}

\end{document}